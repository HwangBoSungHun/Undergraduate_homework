\documentclass[]{article}
\usepackage{lmodern}
\usepackage{amssymb,amsmath}
\usepackage{ifxetex,ifluatex}
\usepackage{fixltx2e} % provides \textsubscript
\ifnum 0\ifxetex 1\fi\ifluatex 1\fi=0 % if pdftex
  \usepackage[T1]{fontenc}
  \usepackage[utf8]{inputenc}
\else % if luatex or xelatex
  \ifxetex
    \usepackage{mathspec}
  \else
    \usepackage{fontspec}
  \fi
  \defaultfontfeatures{Ligatures=TeX,Scale=MatchLowercase}
\fi
% use upquote if available, for straight quotes in verbatim environments
\IfFileExists{upquote.sty}{\usepackage{upquote}}{}
% use microtype if available
\IfFileExists{microtype.sty}{%
\usepackage{microtype}
\UseMicrotypeSet[protrusion]{basicmath} % disable protrusion for tt fonts
}{}
\usepackage[margin=1in]{geometry}
\usepackage{hyperref}
\hypersetup{unicode=true,
            pdftitle={HW2.R},
            pdfauthor={HBSH},
            pdfborder={0 0 0},
            breaklinks=true}
\urlstyle{same}  % don't use monospace font for urls
\usepackage{color}
\usepackage{fancyvrb}
\newcommand{\VerbBar}{|}
\newcommand{\VERB}{\Verb[commandchars=\\\{\}]}
\DefineVerbatimEnvironment{Highlighting}{Verbatim}{commandchars=\\\{\}}
% Add ',fontsize=\small' for more characters per line
\usepackage{framed}
\definecolor{shadecolor}{RGB}{248,248,248}
\newenvironment{Shaded}{\begin{snugshade}}{\end{snugshade}}
\newcommand{\KeywordTok}[1]{\textcolor[rgb]{0.13,0.29,0.53}{\textbf{#1}}}
\newcommand{\DataTypeTok}[1]{\textcolor[rgb]{0.13,0.29,0.53}{#1}}
\newcommand{\DecValTok}[1]{\textcolor[rgb]{0.00,0.00,0.81}{#1}}
\newcommand{\BaseNTok}[1]{\textcolor[rgb]{0.00,0.00,0.81}{#1}}
\newcommand{\FloatTok}[1]{\textcolor[rgb]{0.00,0.00,0.81}{#1}}
\newcommand{\ConstantTok}[1]{\textcolor[rgb]{0.00,0.00,0.00}{#1}}
\newcommand{\CharTok}[1]{\textcolor[rgb]{0.31,0.60,0.02}{#1}}
\newcommand{\SpecialCharTok}[1]{\textcolor[rgb]{0.00,0.00,0.00}{#1}}
\newcommand{\StringTok}[1]{\textcolor[rgb]{0.31,0.60,0.02}{#1}}
\newcommand{\VerbatimStringTok}[1]{\textcolor[rgb]{0.31,0.60,0.02}{#1}}
\newcommand{\SpecialStringTok}[1]{\textcolor[rgb]{0.31,0.60,0.02}{#1}}
\newcommand{\ImportTok}[1]{#1}
\newcommand{\CommentTok}[1]{\textcolor[rgb]{0.56,0.35,0.01}{\textit{#1}}}
\newcommand{\DocumentationTok}[1]{\textcolor[rgb]{0.56,0.35,0.01}{\textbf{\textit{#1}}}}
\newcommand{\AnnotationTok}[1]{\textcolor[rgb]{0.56,0.35,0.01}{\textbf{\textit{#1}}}}
\newcommand{\CommentVarTok}[1]{\textcolor[rgb]{0.56,0.35,0.01}{\textbf{\textit{#1}}}}
\newcommand{\OtherTok}[1]{\textcolor[rgb]{0.56,0.35,0.01}{#1}}
\newcommand{\FunctionTok}[1]{\textcolor[rgb]{0.00,0.00,0.00}{#1}}
\newcommand{\VariableTok}[1]{\textcolor[rgb]{0.00,0.00,0.00}{#1}}
\newcommand{\ControlFlowTok}[1]{\textcolor[rgb]{0.13,0.29,0.53}{\textbf{#1}}}
\newcommand{\OperatorTok}[1]{\textcolor[rgb]{0.81,0.36,0.00}{\textbf{#1}}}
\newcommand{\BuiltInTok}[1]{#1}
\newcommand{\ExtensionTok}[1]{#1}
\newcommand{\PreprocessorTok}[1]{\textcolor[rgb]{0.56,0.35,0.01}{\textit{#1}}}
\newcommand{\AttributeTok}[1]{\textcolor[rgb]{0.77,0.63,0.00}{#1}}
\newcommand{\RegionMarkerTok}[1]{#1}
\newcommand{\InformationTok}[1]{\textcolor[rgb]{0.56,0.35,0.01}{\textbf{\textit{#1}}}}
\newcommand{\WarningTok}[1]{\textcolor[rgb]{0.56,0.35,0.01}{\textbf{\textit{#1}}}}
\newcommand{\AlertTok}[1]{\textcolor[rgb]{0.94,0.16,0.16}{#1}}
\newcommand{\ErrorTok}[1]{\textcolor[rgb]{0.64,0.00,0.00}{\textbf{#1}}}
\newcommand{\NormalTok}[1]{#1}
\usepackage{graphicx,grffile}
\makeatletter
\def\maxwidth{\ifdim\Gin@nat@width>\linewidth\linewidth\else\Gin@nat@width\fi}
\def\maxheight{\ifdim\Gin@nat@height>\textheight\textheight\else\Gin@nat@height\fi}
\makeatother
% Scale images if necessary, so that they will not overflow the page
% margins by default, and it is still possible to overwrite the defaults
% using explicit options in \includegraphics[width, height, ...]{}
\setkeys{Gin}{width=\maxwidth,height=\maxheight,keepaspectratio}
\IfFileExists{parskip.sty}{%
\usepackage{parskip}
}{% else
\setlength{\parindent}{0pt}
\setlength{\parskip}{6pt plus 2pt minus 1pt}
}
\setlength{\emergencystretch}{3em}  % prevent overfull lines
\providecommand{\tightlist}{%
  \setlength{\itemsep}{0pt}\setlength{\parskip}{0pt}}
\setcounter{secnumdepth}{0}
% Redefines (sub)paragraphs to behave more like sections
\ifx\paragraph\undefined\else
\let\oldparagraph\paragraph
\renewcommand{\paragraph}[1]{\oldparagraph{#1}\mbox{}}
\fi
\ifx\subparagraph\undefined\else
\let\oldsubparagraph\subparagraph
\renewcommand{\subparagraph}[1]{\oldsubparagraph{#1}\mbox{}}
\fi

%%% Use protect on footnotes to avoid problems with footnotes in titles
\let\rmarkdownfootnote\footnote%
\def\footnote{\protect\rmarkdownfootnote}

%%% Change title format to be more compact
\usepackage{titling}

% Create subtitle command for use in maketitle
\newcommand{\subtitle}[1]{
  \posttitle{
    \begin{center}\large#1\end{center}
    }
}

\setlength{\droptitle}{-2em}
  \title{HW2.R}
  \pretitle{\vspace{\droptitle}\centering\huge}
  \posttitle{\par}
  \author{HBSH}
  \preauthor{\centering\large\emph}
  \postauthor{\par}
  \predate{\centering\large\emph}
  \postdate{\par}
  \date{Wed Apr 11 10:42:36 2018}


\begin{document}
\maketitle

\begin{Shaded}
\begin{Highlighting}[]
\KeywordTok{library}\NormalTok{(dplyr)}
\end{Highlighting}
\end{Shaded}

\begin{verbatim}
## 
## Attaching package: 'dplyr'
\end{verbatim}

\begin{verbatim}
## The following objects are masked from 'package:stats':
## 
##     filter, lag
\end{verbatim}

\begin{verbatim}
## The following objects are masked from 'package:base':
## 
##     intersect, setdiff, setequal, union
\end{verbatim}

\begin{Shaded}
\begin{Highlighting}[]
\KeywordTok{dev.off}\NormalTok{()}
\end{Highlighting}
\end{Shaded}

\begin{verbatim}
## null device 
##           1
\end{verbatim}

\begin{Shaded}
\begin{Highlighting}[]
\NormalTok{## Part1}
\KeywordTok{setwd}\NormalTok{(}\StringTok{"~/Desktop/3-1/DataMining/HomeWork/HW2"}\NormalTok{)}
\NormalTok{train_data <-}\StringTok{ }\KeywordTok{read.csv}\NormalTok{(}\StringTok{"HWtrain.data.txt"}\NormalTok{, }\DataTypeTok{header =} \OtherTok{FALSE}\NormalTok{)}
\NormalTok{test_data <-}\StringTok{ }\KeywordTok{read.csv}\NormalTok{(}\StringTok{"HWtest.data.txt"}\NormalTok{, }\DataTypeTok{header =} \OtherTok{FALSE}\NormalTok{)}

\NormalTok{train <-}\StringTok{ }\NormalTok{train_data[, }\OperatorTok{-}\DecValTok{1}\NormalTok{]}
\NormalTok{train_y <-train_data[, }\DecValTok{1}\NormalTok{]}
\NormalTok{test <-}\StringTok{ }\NormalTok{test_data[, }\OperatorTok{-}\DecValTok{1}\NormalTok{]}

\NormalTok{train_norm <-}\StringTok{ }\KeywordTok{scale}\NormalTok{(train)}

\NormalTok{train.pr <-}\StringTok{ }\KeywordTok{princomp}\NormalTok{(train_norm)}
\KeywordTok{summary}\NormalTok{(train.pr)}
\end{Highlighting}
\end{Shaded}

\begin{verbatim}
## Importance of components:
##                           Comp.1    Comp.2    Comp.3     Comp.4     Comp.5
## Standard deviation     2.1638746 1.5701494 1.2014961 0.94633050 0.93237813
## Proportion of Variance 0.3622751 0.1907464 0.1116912 0.06928831 0.06726025
## Cumulative Proportion  0.3622751 0.5530215 0.6647127 0.73400099 0.80126124
##                            Comp.6     Comp.7     Comp.8     Comp.9
## Standard deviation     0.80021359 0.74072762 0.58671307 0.53958380
## Proportion of Variance 0.04954344 0.04245134 0.02663335 0.02252642
## Cumulative Proportion  0.85080468 0.89325601 0.91988937 0.94241578
##                           Comp.10    Comp.11    Comp.12     Comp.13
## Standard deviation     0.49965421 0.47404359 0.40897252 0.320370897
## Proportion of Variance 0.01931583 0.01738645 0.01294084 0.007941095
## Cumulative Proportion  0.96173161 0.97911806 0.99205890 1.000000000
\end{verbatim}

\begin{Shaded}
\begin{Highlighting}[]
\CommentTok{#Creating a biplot}
\KeywordTok{biplot}\NormalTok{(train.pr)}

\CommentTok{#Getting proportion of variance for a scree plot}
\NormalTok{train.pr.var <-}\StringTok{ }\NormalTok{train.pr}\OperatorTok{$}\NormalTok{sdev}\OperatorTok{^}\DecValTok{2}
\NormalTok{pve <-}\StringTok{ }\NormalTok{train.pr.var }\OperatorTok{/}\StringTok{ }\KeywordTok{sum}\NormalTok{(train.pr.var)}
\KeywordTok{plot}\NormalTok{(pve, }\DataTypeTok{xlab =} \StringTok{"Principal Component"}\NormalTok{, }
     \DataTypeTok{ylab =} \StringTok{"Portion of Variance Explaine"}\NormalTok{,}
     \DataTypeTok{ylim =} \KeywordTok{c}\NormalTok{(}\DecValTok{0}\NormalTok{, }\DecValTok{1}\NormalTok{), }\DataTypeTok{type =} \StringTok{"b"}\NormalTok{)}

\CommentTok{# (a) When we use the first two principle components, }
\CommentTok{# how much information of the total do we use? }
\KeywordTok{sprintf}\NormalTok{(}\StringTok{"%.2f%%"}\NormalTok{, }\KeywordTok{sum}\NormalTok{(pve[}\DecValTok{1}\OperatorTok{:}\DecValTok{2}\NormalTok{]) }\OperatorTok{*}\StringTok{ }\DecValTok{100}\NormalTok{)}
\end{Highlighting}
\end{Shaded}

\begin{verbatim}
## [1] "55.30%"
\end{verbatim}

\begin{Shaded}
\begin{Highlighting}[]
\CommentTok{# (b) Find the top-three original variables among }
\CommentTok{# the 13 that contributed to the construction }
\CommentTok{# of the first principle component. }
\KeywordTok{abs}\NormalTok{(train.pr}\OperatorTok{$}\NormalTok{loadings[, }\DecValTok{1}\NormalTok{]) }\OperatorTok\StringTok{ }
\StringTok{  }\KeywordTok{sort}\NormalTok{(}\DataTypeTok{decreasing =} \OtherTok{TRUE}\NormalTok{) }\OperatorTok
\StringTok{  }\KeywordTok{head}\NormalTok{(}\DecValTok{3}\NormalTok{)}
\end{Highlighting}
\end{Shaded}

\begin{verbatim}
##        V2       V12        V7 
## 0.4227611 0.3934570 0.3793968
\end{verbatim}

\begin{Shaded}
\begin{Highlighting}[]
\CommentTok{# (c) Draw a plot of the 1st scores (on the X axis) and the 2nd scores }
\CommentTok{# (on the Y axis) of the descriptive values in HWtrain.data.txt together}
\CommentTok{# with the wine qualities and similarly transformed score values of the }
\CommentTok{# five future observations in HWtest.data.txt.  Then try to predict the }
\CommentTok{# wine quality of each of the five future observations.}

\KeywordTok{plot}\NormalTok{(train.pr}\OperatorTok{$}\NormalTok{scores[,}\DecValTok{1}\NormalTok{], train.pr}\OperatorTok{$}\NormalTok{scores[,}\DecValTok{2}\NormalTok{], }\DataTypeTok{pch =} \DecValTok{20}\NormalTok{,}
     \DataTypeTok{xlab =} \StringTok{"PC1"}\NormalTok{, }\DataTypeTok{ylab =} \StringTok{"PC2"}\NormalTok{, }\DataTypeTok{col =}\NormalTok{ train_y, }\DataTypeTok{main =} \StringTok{"Wine quality "}\NormalTok{)}
\KeywordTok{text}\NormalTok{(train.pr}\OperatorTok{$}\NormalTok{scores[,}\DecValTok{1}\NormalTok{], train.pr}\OperatorTok{$}\NormalTok{scores[,}\DecValTok{2}\NormalTok{], }
     \DataTypeTok{labels =}\NormalTok{ train_y, }\DataTypeTok{pos =} \DecValTok{4}\NormalTok{, }\DataTypeTok{col =}\NormalTok{ train_y, }\DataTypeTok{cex =} \FloatTok{0.6}\NormalTok{)}

\NormalTok{test_norm <-}\StringTok{ }\KeywordTok{scale}\NormalTok{(test)}
\NormalTok{test.pr <-}\StringTok{ }\KeywordTok{as.matrix}\NormalTok{(test_norm) }\OperatorTok\StringTok{ }\KeywordTok{as.matrix}\NormalTok{(train.pr}\OperatorTok{$}\NormalTok{loadings[, }\DecValTok{1}\OperatorTok{:}\DecValTok{2}\NormalTok{])}
\KeywordTok{points}\NormalTok{(test.pr[,}\DecValTok{1}\NormalTok{], test.pr[,}\DecValTok{2}\NormalTok{], }\DataTypeTok{col =} \StringTok{"blue"}\NormalTok{, }\DataTypeTok{pch =}\DecValTok{15}\NormalTok{)}
\KeywordTok{text}\NormalTok{(test.pr[,}\DecValTok{1}\NormalTok{], test.pr[,}\DecValTok{2}\NormalTok{],  }
     \DataTypeTok{labels =} \DecValTok{1}\OperatorTok{:}\DecValTok{5}\NormalTok{, }\DataTypeTok{pos =} \DecValTok{4}\NormalTok{, }\DataTypeTok{col =} \StringTok{"blue"}\NormalTok{, }\DataTypeTok{cex =} \FloatTok{1.2}\NormalTok{)}
\KeywordTok{png}\NormalTok{(}\StringTok{"predictWineQuality.png"}\NormalTok{, }\DataTypeTok{width=}\DecValTok{400}\NormalTok{, }\DataTypeTok{height=}\DecValTok{300}\NormalTok{, }\DataTypeTok{pointsize=}\DecValTok{13}\NormalTok{)}
\KeywordTok{dev.off}\NormalTok{()}
\end{Highlighting}
\end{Shaded}

\begin{verbatim}
## pdf 
##   2
\end{verbatim}

\begin{Shaded}
\begin{Highlighting}[]
\NormalTok{## Part2}
\KeywordTok{library}\NormalTok{(readxl)}
\CommentTok{#library(caret)}

\NormalTok{data2 <-}\StringTok{ }\KeywordTok{read_xlsx}\NormalTok{(}\StringTok{"HW2_data_2.xlsx"}\NormalTok{)}
\NormalTok{data2}\OperatorTok{$}\NormalTok{Y <-}\StringTok{ }\KeywordTok{as.factor}\NormalTok{(data2}\OperatorTok{$}\NormalTok{Y)}
\NormalTok{data2_X <-}\StringTok{ }\NormalTok{data2}\OperatorTok{$}\NormalTok{Xun}
\NormalTok{data2_norm_X <-}\StringTok{ }\NormalTok{(data2_X }\OperatorTok{-}\StringTok{ }\KeywordTok{min}\NormalTok{(data2_X)) }\OperatorTok{/}\StringTok{ }\NormalTok{(}\KeywordTok{max}\NormalTok{(data2_X) }\OperatorTok{-}\StringTok{ }\KeywordTok{min}\NormalTok{(data2_X))}
\NormalTok{data2_norm <-}\StringTok{ }\NormalTok{data2}
\NormalTok{data2_norm}\OperatorTok{$}\NormalTok{Xun <-}\StringTok{ }\NormalTok{data2_norm_X}

\NormalTok{data2_norm_}\DecValTok{1}\NormalTok{ <-}\StringTok{ }\NormalTok{data2_norm }\CommentTok{# cut-off value 0.4}
\NormalTok{data2_norm_}\DecValTok{1}\OperatorTok{$}\NormalTok{Y <-}\StringTok{ }\KeywordTok{as.factor}\NormalTok{(}\KeywordTok{ifelse}\NormalTok{(data2_norm}\OperatorTok{$}\NormalTok{Xun }\OperatorTok{>=}\StringTok{ }\FloatTok{0.4}\NormalTok{, }\DecValTok{1}\NormalTok{, }\DecValTok{0}\NormalTok{)) }

\NormalTok{data2_norm_}\DecValTok{2}\NormalTok{ <-}\StringTok{ }\NormalTok{data2_norm }\CommentTok{# cut-off value 0.5}
\NormalTok{data2_norm_}\DecValTok{2}\OperatorTok{$}\NormalTok{Y <-}\StringTok{ }\KeywordTok{as.factor}\NormalTok{(}\KeywordTok{ifelse}\NormalTok{(data2_norm}\OperatorTok{$}\NormalTok{Xun }\OperatorTok{>=}\StringTok{ }\FloatTok{0.5}\NormalTok{, }\DecValTok{1}\NormalTok{, }\DecValTok{0}\NormalTok{)) }

\NormalTok{data2_norm_}\DecValTok{3}\NormalTok{ <-}\StringTok{ }\NormalTok{data2_norm }\CommentTok{# cut-off value 0.4}
\NormalTok{data2_norm_}\DecValTok{3}\OperatorTok{$}\NormalTok{Y <-}\StringTok{ }\KeywordTok{as.factor}\NormalTok{(}\KeywordTok{ifelse}\NormalTok{(data2_norm}\OperatorTok{$}\NormalTok{Xun }\OperatorTok{>=}\StringTok{ }\FloatTok{0.6}\NormalTok{, }\DecValTok{1}\NormalTok{, }\DecValTok{0}\NormalTok{)) }

\CommentTok{#confusionMatrix(data2_norm_1$Y, data2_norm$Y)}
\CommentTok{#confusionMatrix(data2_norm_2$Y, data2_norm$Y)}
\CommentTok{#confusionMatrix(data2_norm_3$Y, data2_norm$Y)}
\NormalTok{confusionMat <-}\StringTok{ }\ControlFlowTok{function}\NormalTok{(pred, actual)}
\NormalTok{\{}
\NormalTok{  TP <-}\StringTok{ }\KeywordTok{sum}\NormalTok{(pred}\OperatorTok{$}\NormalTok{Y }\OperatorTok{==}\StringTok{ }\DecValTok{1} \OperatorTok{&}\StringTok{ }\NormalTok{actual}\OperatorTok{$}\NormalTok{Y }\OperatorTok{==}\StringTok{ }\DecValTok{1}\NormalTok{)}
\NormalTok{  FP <-}\StringTok{ }\KeywordTok{sum}\NormalTok{(pred}\OperatorTok{$}\NormalTok{Y }\OperatorTok{==}\StringTok{ }\DecValTok{1} \OperatorTok{&}\StringTok{ }\NormalTok{actual}\OperatorTok{$}\NormalTok{Y }\OperatorTok{==}\StringTok{ }\DecValTok{0}\NormalTok{)}
\NormalTok{  FN <-}\StringTok{ }\KeywordTok{sum}\NormalTok{(pred}\OperatorTok{$}\NormalTok{Y }\OperatorTok{==}\StringTok{ }\DecValTok{0} \OperatorTok{&}\StringTok{ }\NormalTok{actual}\OperatorTok{$}\NormalTok{Y }\OperatorTok{==}\StringTok{ }\DecValTok{1}\NormalTok{)}
\NormalTok{  TN <-}\StringTok{ }\KeywordTok{sum}\NormalTok{(pred}\OperatorTok{$}\NormalTok{Y }\OperatorTok{==}\StringTok{ }\DecValTok{0} \OperatorTok{&}\StringTok{ }\NormalTok{actual}\OperatorTok{$}\NormalTok{Y }\OperatorTok{==}\StringTok{ }\DecValTok{0}\NormalTok{)}
  
\NormalTok{  confuMat <-}\StringTok{ }\KeywordTok{matrix}\NormalTok{(}\KeywordTok{c}\NormalTok{(TP, FP, FN, TN), }\DataTypeTok{byrow =} \OtherTok{TRUE}\NormalTok{, }\DataTypeTok{nrow =} \DecValTok{2}\NormalTok{, }\DataTypeTok{ncol =} \DecValTok{2}\NormalTok{)}
\NormalTok{  confuMat_df <-}\StringTok{ }\KeywordTok{data.frame}\NormalTok{(confuMat, }\DataTypeTok{row.names =} \KeywordTok{c}\NormalTok{(}\StringTok{"pred_True(1)"}\NormalTok{, }\StringTok{"pred_False(0)"}\NormalTok{))}
  \KeywordTok{colnames}\NormalTok{(confuMat_df) <-}\StringTok{ }\KeywordTok{c}\NormalTok{(}\StringTok{"act_True(1)"}\NormalTok{, }\StringTok{"act_False(0)"}\NormalTok{)}
  
\NormalTok{  confuMat_df}
  
\NormalTok{  accur <-}\StringTok{ }\NormalTok{(TP }\OperatorTok{+}\StringTok{ }\NormalTok{TN) }\OperatorTok{/}\StringTok{ }\NormalTok{(TP }\OperatorTok{+}\StringTok{ }\NormalTok{FP }\OperatorTok{+}\StringTok{ }\NormalTok{FN }\OperatorTok{+}\StringTok{ }\NormalTok{TN)}
\NormalTok{  sensi <-}\StringTok{ }\NormalTok{TP }\OperatorTok{/}\StringTok{ }\NormalTok{(TP }\OperatorTok{+}\StringTok{ }\NormalTok{FN)}
\NormalTok{  speci <-}\StringTok{ }\NormalTok{TN }\OperatorTok{/}\StringTok{ }\NormalTok{(FP }\OperatorTok{+}\StringTok{ }\NormalTok{TN)}
\NormalTok{  preci <-}\StringTok{ }\NormalTok{TP }\OperatorTok{/}\StringTok{ }\NormalTok{(TP }\OperatorTok{+}\StringTok{ }\NormalTok{FP)}
\NormalTok{  result.list <-}\StringTok{ }\KeywordTok{list}\NormalTok{(}\DataTypeTok{ConfusionMatrix =}\NormalTok{ confuMat_df, }\DataTypeTok{Accuracy =}\NormalTok{ accur, }
                      \DataTypeTok{Sensitivity =}\NormalTok{ sensi, }\DataTypeTok{Specificity =}\NormalTok{ speci, }
                      \DataTypeTok{Precision =}\NormalTok{ preci) }
  \KeywordTok{return}\NormalTok{(result.list)}
\NormalTok{\}}

\KeywordTok{confusionMat}\NormalTok{(data2_norm_}\DecValTok{1}\NormalTok{, data2_norm)}
\end{Highlighting}
\end{Shaded}

\begin{verbatim}
## $ConfusionMatrix
##               act_True(1) act_False(0)
## pred_True(1)            8            3
## pred_False(0)           0            4
## 
## $Accuracy
## [1] 0.8
## 
## $Sensitivity
## [1] 1
## 
## $Specificity
## [1] 0.5714286
## 
## $Precision
## [1] 0.7272727
\end{verbatim}

\begin{Shaded}
\begin{Highlighting}[]
\KeywordTok{confusionMat}\NormalTok{(data2_norm_}\DecValTok{2}\NormalTok{, data2_norm)}
\end{Highlighting}
\end{Shaded}

\begin{verbatim}
## $ConfusionMatrix
##               act_True(1) act_False(0)
## pred_True(1)            7            2
## pred_False(0)           1            5
## 
## $Accuracy
## [1] 0.8
## 
## $Sensitivity
## [1] 0.875
## 
## $Specificity
## [1] 0.7142857
## 
## $Precision
## [1] 0.7777778
\end{verbatim}

\begin{Shaded}
\begin{Highlighting}[]
\KeywordTok{confusionMat}\NormalTok{(data2_norm_}\DecValTok{3}\NormalTok{, data2_norm)}
\end{Highlighting}
\end{Shaded}

\begin{verbatim}
## $ConfusionMatrix
##               act_True(1) act_False(0)
## pred_True(1)            4            0
## pred_False(0)           4            7
## 
## $Accuracy
## [1] 0.7333333
## 
## $Sensitivity
## [1] 0.5
## 
## $Specificity
## [1] 1
## 
## $Precision
## [1] 1
\end{verbatim}


\end{document}
